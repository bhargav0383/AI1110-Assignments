%%%%%%%%%%%%%%%%%%%%%%%%%%%%%%%%%%%%%%%%%%%%%%%%%%%%%%%%%%%%%%%
%	
% Welcome to Overleaf --- just edit your LaTeX on the left,
% and we'll compile it for you on the right. If you open the
% 'Share' menu, you can invite other users to edit at the same
% time. See www.overleaf.com/learn for more info. Enjoy!
%
%%%%%%%%%%%%%%%%%%%%%%%%%%%%%%%%%%%%%%%%%%%%%%%%%%%%%%%%%%%%%%%

% Inbuilt themes in beamer
\documentclass{beamer}

%packages:
% \usepackage{tfrupee}
% \usepackage{amsmath}
% \usepackage{amssymb}
% \usepackage{gensymb}
% \usepackage{txfonts}

% \def\inputGnumericTable{}

% \usepackage[latin1]{inputenc}                                 
% \usepackage{color}                                            
% \usepackage{array}                                            
% \usepackage{longtable}                                        
% \usepackage{calc}                                             
% \usepackage{multirow}                                         
% \usepackage{hhline}                                           
% \usepackage{ifthen}
% \usepackage{caption} 
% \captionsetup[table]{skip=3pt}  
% \providecommand{\pr}[1]{\ensuremath{\Pr\left(#1\right)}}
% \providecommand{\cbrak}[1]{\ensuremath{\left\{#1\right\}}}
% %\renewcommand{\thefigure}{\arabic{table}}
% \renewcommand{\thetable}{\arabic{table}}      

\setbeamertemplate{caption}[numbered]{}

\usepackage{enumitem}
\usepackage{tfrupee}
\usepackage{amsmath}
\usepackage{amssymb}
\usepackage{gensymb}
\usepackage{graphicx}
\usepackage{txfonts}

\def\inputGnumericTable{}

\usepackage[latin1]{inputenc}                                 
\usepackage{color}    
\usepackage{textcomp, gensymb}         
\usepackage{array}                                            
\usepackage{longtable}                                        
\usepackage{calc}                                             
\usepackage{multirow}                                         
\usepackage{hhline}                             
\usepackage{mathtools}
\usepackage{ifthen}
\usepackage{caption} 
\providecommand{\pr}[1]{\ensuremath{\Pr\left(#1\right)}}
\providecommand{\cbrak}[1]{\ensuremath{\left\{#1\right\}}}
\renewcommand{\thefigure}{\arabic{table}}
\renewcommand{\thetable}{\arabic{table}}   
\providecommand{\brak}[1]{\ensuremath{\left(#1\right)}}

% Theme choice:
\usetheme{CambridgeUS}

% Title page details: 
\title{AI1110 \\ Assignment-9} 
\author{Rajulapati Bhargava Ram \\ CS21BTECH11052}
\date{\today}
\logo{\large \LaTeX{}}


\begin{document}

% Title page frame
\begin{frame}
    \titlepage 
\end{frame}
\logo{}


% Outline frame
\begin{frame}{Outline}
    \tableofcontents
\end{frame}



\section{Question}
\begin{frame}{Question}
    \begin{block}{\textbf{Papoullis 5-32:} } 
       \begin{enumerate}[label=(\alph*)]
          \item Show that if $m$ is the median of $X$, then
              \begin{align}
                 E\{|x-a|\} = E\{|x-m|\} + 2\int_a^m \! (x-a)f(x) \, \mathrm{d}x  
              \end{align}
              for any $a$.
          \item Find $c$ such that $E\{|x-c|\}$ is minimum.
       \end{enumerate}
     \end{block}
     
\end{frame}


\section{Definitions}
\begin{frame}{Definitions}
   \begin{block}{Cumulative Distribution Function}
      If $f_X(x)$ is probability density function of a random variable X, then it's corresponding Probability (cumulative) distribution function $F_X(x)$ is 
     \begin{align}
        F_X(x) = P\cbrak{X \le x}
              &= \int_{-\infty}^{x} f_X(x) \, dx \label{eq2} 
     \end{align}
   \end{block}	

	\begin{block}{Expected Value}
	   The expected value or mean of a random variable X is by definition integral
	   \begin{align}
	      E\{X\} = \int_{-\infty}^{\infty} xf_X(x) \, dx \label{eq3}
	   \end{align}
	\end{block}

\end{frame}



\section{Solution}
\begin{frame}{Solution}
  From the given information;
       \begin{align}
         \frac{\partial |x-a|}{\partial a}=
            \begin{dcases}
                 1 & x < a \\
                -1 & x \geq a \\
            \end{dcases}
       \end{align} 
  From definition we know that,
      \begin{align}
         E\{|X-a|\} = \int_{-\infty}^{\infty} |x-a|f_X(x) \, dx
      \end{align}
\end{frame}


\begin{frame}
  Let $I(a) = E\{|X-a|\}$ then, the partial derivative of $I(a)$ w.r.t to $a$  $(I'(a))$ is
   \begin{align}
      \frac{\partial I(a)}{\partial a} &= \frac{\partial \{\int_{-\infty}^{\infty} |x-a|f_X(x) \, dx\}}{\partial a} \\
                                       &= \int_{-\infty}^{\infty} \frac{\partial |x-a|}{\partial a}f_X(x) \, dx \\
                                       &= \int_{-\infty}^{a} \frac{\partial |x-a|}{\partial a}f_X(x) \, dx + \int_{a}^{\infty} \frac{\partial |x-a|}{\partial a}f_X(x) \, dx \\
                                       &= \int_{-\infty}^{a} (1)f_X(x) \, dx + \int_{a}^{\infty} (-1)f_X(x) \, dx \\
                                       &=  P\cbrak{X \le a} - P\cbrak{X > a} \\
                     \therefore I'(a)  &= 2 F_X(a) - 1 
   \end{align}    
\end{frame}
 

\begin{frame}{Part (a)}
   Start with
   \begin{align}
      \int_{m}^{a} I'(\alpha) \, d\alpha &= I(a) - I(m) \\
      							   I(a) &= I(m) + \int_{m}^{a} [2 F_X(\alpha) - 1]  \, d\alpha \\
      							        &= I(m) + 2\int_{m}^{a}  F_X(\alpha)   \, d\alpha - \int_{m}^{a} 1  \, d\alpha \\
      							        &= I(m) + 2\int_{m}^{a}  F_X(\alpha)   \, d\alpha - (a - m) \label{eq15}
   \end{align}    
    
    Where, replace $\alpha$ with $x$ in the integral below
    \begin{align}
       \int_{m}^{a} F_X(\alpha)  \, d\alpha &= a F_X(a) - m F_X(m) - \int_{m}^{a} x f_X(x)   \, dx  \label{eq16}
    \end{align}
\end{frame}


\begin{frame}{}
   As $m$ is the median of $X$, C.M.F $F_X(m) = \frac{1}{2}$ 
   
   On substituting equation \eqref{eq16} in equation \eqref{eq15} we get,
   \begin{align}
      I(a) &= I(m) + 2 \cbrak{a F_X(a) - m F_X(m) - \int_{m}^{a} x f_X(x)   \, dx} - (a - m) \\
           &= I(m) + 2\int_{a}^{m} x f_X(x)   \, dx + 2 \cbrak{a F_X(a) - m(\frac{1}{2})} + m - a \\
           &= I(m) + 2\int_{a}^{m} x f_X(x)   \, dx + a( 2F_X(a) - 1) \\
           &= I(m) + 2\int_{a}^{m} x f_X(x)   \, dx + 2a( F_X(a) - F_X(m)) \\
           &= I(m) + 2\int_{a}^{m} x f_X(x)   \, dx + 2a \int_{m}^{a} f_X(x) \, dx  \label{eq21} \\
           &= I(m) + 2\int_{a}^{m} (x - a) f_X(x)   \, dx
   \end{align}
\end{frame}


\begin{frame}{}
   We have used this in above equation \eqref{eq21},
   \begin{align}
      \int_{m}^{a} f_X(x) \, dx &= \int_{-\infty}^{a} f_X(x) \, dx - \int_{-\infty}^{m} f_X(x) \, dx  \\
      						   &= F_X(a) - F_X(m)
   \end{align}
   
   \begin{align}
      \therefore I(a) &= I(m) + 2\int_{a}^{m} (x - a) f_X(x)   \, dx \\
           E\{|X-a|\} &= E\{|X-m|\} + 2\int_{a}^{m} (x - a) f_X(x)   \, dx
   \end{align}
   Hence proved, If $m$ is the median of $X$ then the above equation is true for any $a$.
\end{frame}


\begin{frame}{Part (b)}
   The value of $c$ for which $E\{|X-c|\}$ is minimum can be found by,
   
   $I(c) = E\{|X-c|\}$ is minimum if,
   \begin{align}
     I'(c) &= 0 \\
     2 F_X(c) - 1 &= 0 \\
     \therefore F_X(c) &= \frac{1}{2} \\
          \therefore c &= m 
   \end{align}
    Hence, the value of $c$ is $m$ which is the median of $X$.
\end{frame}

\end{document}