%%%%%%%%%%%%%%%%%%%%%%%%%%%%%%%%%%%%%%%%%%%%%%%%%%%%%%%%%%%%%%%
%	
% Welcome to Overleaf --- just edit your LaTeX on the left,
% and we'll compile it for you on the right. If you open the
% 'Share' menu, you can invite other users to edit at the same
% time. See www.overleaf.com/learn for more info. Enjoy!
%
%%%%%%%%%%%%%%%%%%%%%%%%%%%%%%%%%%%%%%%%%%%%%%%%%%%%%%%%%%%%%%%

% Inbuilt themes in beamer
\documentclass{beamer}

%packages:
% \usepackage{tfrupee}
% \usepackage{amsmath}
% \usepackage{amssymb}
% \usepackage{gensymb}
% \usepackage{txfonts}

% \def\inputGnumericTable{}

% \usepackage[latin1]{inputenc}                                 
% \usepackage{color}                                            
% \usepackage{array}                                            
% \usepackage{longtable}                                        
% \usepackage{calc}                                             
% \usepackage{multirow}                                         
% \usepackage{hhline}                                           
% \usepackage{ifthen}
% \usepackage{caption} 
% \captionsetup[table]{skip=3pt}  
% \providecommand{\pr}[1]{\ensuremath{\Pr\left(#1\right)}}
% \providecommand{\cbrak}[1]{\ensuremath{\left\{#1\right\}}}
% %\renewcommand{\thefigure}{\arabic{table}}
% \renewcommand{\thetable}{\arabic{table}}      

\setbeamertemplate{caption}[numbered]{}

\usepackage{enumitem}
\usepackage{tfrupee}
\usepackage{amsmath}
\usepackage{amssymb}
\usepackage{gensymb}
\usepackage{graphicx}
\usepackage{txfonts}

\def\inputGnumericTable{}

\usepackage[latin1]{inputenc}                                 
\usepackage{color}    
\usepackage{textcomp, gensymb}         
\usepackage{array}                                            
\usepackage{longtable}                                        
\usepackage{calc}                                             
\usepackage{multirow}                                         
\usepackage{hhline}                             
\usepackage{mathtools}
\usepackage{ifthen}
\usepackage{caption} 
\providecommand{\pr}[1]{\ensuremath{\Pr\left(#1\right)}}
\providecommand{\cbrak}[1]{\ensuremath{\left\{#1\right\}}}
\renewcommand{\thefigure}{\arabic{table}}
\renewcommand{\thetable}{\arabic{table}}   
\providecommand{\brak}[1]{\ensuremath{\left(#1\right)}}

% Theme choice:
\usetheme{CambridgeUS}

% Title page details: 
\title{AI1110 \\ Assignment-10} 
\author{Rajulapati Bhargava Ram \\ CS21BTECH11052}
\date{\today}
\logo{\large \LaTeX{}}


\begin{document}

% Title page frame
\begin{frame}
    \titlepage 
\end{frame}
\logo{}


% Outline frame
\begin{frame}{Outline}
    \tableofcontents
\end{frame}



\section{Question}
\begin{frame}{Question}
    \begin{block}{\textbf{Papoullis 6-75:} } 
      The random variable $x$ has a student $t$ distribution $t(n)$. Show that 
	\begin{align}
	   E\cbrak{x^2}=\frac{n}{n-2}
	\end{align}	      
  
     \end{block}
     
\end{frame}


\section{Theory}
\begin{frame}{Theory}
   \begin{block}{Student $t$ Distribution}
      A random variable $x$ has a Student $t$ Distribution $t(n)$ with n degrees of freedom if for $-\infty<X<\infty$. Here $y,z$ are two independent R.V's
     \begin{align}
       x^2 = \frac{n y^2}{z}
     \end{align}
  \end{block}   
  
  \begin{block}{}
     Where $y$ is $N(0,1)$ is Normal random variable for which  
     \begin{align}
       f_X(x) = \frac{1}{\sigma \sqrt{2 \pi}} e^{\frac{- y^2}{2 \sigma^2}}
     \end{align}
     then,
     \begin{align}
       E\cbrak{y^n} =
       \begin{dcases}
                 0 & n = 2k+1 \\
                1.3...(n-1) \sigma^n & n = 2k  \\
       \end{dcases} \label{eq4}
     \end{align}
   \end{block}	
	
\end{frame}

\begin{frame}
   \begin{block}{}
      and $z$ is $\chi^2(n)$ which is CHI-SQUARE Distribution with n degrees of freedom if
      \begin{align}
         f_X(x) =
          \begin{dcases}
                 \frac{x^{n/2 -1}}{2^{n/2} \Gamma\brak{n/2}} e^{- x/2} & x \ge 0 \\
                  0  & otherwise  \\
          \end{dcases} \label{eq5}
	  \end{align}       
   \end{block}
   
   \begin{block}{}
      Where $\Gamma(\alpha)$ represents the gamma function defined as
      \begin{align}
         \Gamma(\alpha) =  \int_{0}^{\infty} x^{\alpha-1} e^{-x} \, dx
      \end{align}
      Where $\alpha$ is an integer, by using integration by parts we get
      \begin{align}
         \Gamma(\alpha) = (n-1) \Gamma(n-1) 
         			   &= (n-1)!
      \end{align}
   \end{block}
   
\end{frame}

\section{Solution}
\begin{frame}{Solution}
   \begin{align}
      E\cbrak{x^2} &= E\cbrak{\frac{n y^2}{z}} \\
      E\cbrak{x^2} &= n \, E\cbrak{y^2} \, E\cbrak{\frac{1}{z}} \label{eq9}
   \end{align}
  So, first we will find $E\cbrak{y^2}$ using above equation \eqref{eq4}. Here $n=2$ so n is even,(given $\sigma = 1$)
  \begin{align}
     E\cbrak{y^2} &= 1.\sigma^2 \\
     \therefore E\cbrak{y^2} &= 1 \label{eq11}
  \end{align}
      
\end{frame}


\begin{frame}
   Now for $E\cbrak{\frac{1}{z}}$ using  equation \eqref{eq5} in moment generating function we get,
   \begin{align}
      E\cbrak{\frac{1}{z}} &= \int_{-\infty}^{\infty} \frac{1}{z} \, f_z(z) \, dz \\
      					  &= \int_{-\infty}^{0} \frac{1}{z} \cbrak{0} \, dx + \int_{0}^{\infty} \frac{1}{z} \cbrak{\frac{z^{n/2 -1}}{2^{n/2} \Gamma(n/2)}e^{- z/2}} \, dz \\
      					  &= \frac{1}{2^{n/2} \Gamma(n/2)}\int_{0}^{\infty} z^{n/2 -2} \, e^{- z/2} \, dz \label{eq14}				  
   \end{align}
   replace $z/2$ with $v$, and use $dz = 2 \, dv$ we get
   \begin{align}
      \int_{0}^{\infty} z^{n/2 -2} e^{- z/2} \, dz 
         &= \int_{0}^{\infty} {(2v)}^{n/2 -2} \, e^{- v} \,  2 \, dv \\
         &= 2^{n/2 - 1} \int_{0}^{\infty} v^{n/2 -2} \, e^{- v} \, dv \label{eq16}
   \end{align}
       
\end{frame}
 

\begin{frame}{}
   On substituting equation \eqref{eq16} in equation \eqref{eq14} we get,
   \begin{align}
      E\cbrak{\frac{1}{z}} 
           &= \frac{1}{2^{n/2} \Gamma(n/2)} \cbrak{2^{n/2 - 1} \int_{0}^{\infty} v^{n/2 -2} e^{- v} \, dv} \\
           &= \frac{2^{n/2 - 1}}{2^{n/2} \Gamma(n/2)} \int_{0}^{\infty} v^{n/2 -2} e^{- v} \, dv
   \end{align}    
   In above equation the integration looks like gamma function, which is $\Gamma(n/2-1)$ then,
   \begin{align}
       E\cbrak{\frac{1}{z}} = \frac{ \Gamma(n/2 - 1)}{2 \; \Gamma(n/2)}
        					   &= \frac{(n/2 - 2)!}{2 \; (n/2 - 1)!} \\
       			\therefore E\cbrak{\frac{1}{z}}   &= \frac{1}{n - 2} \label{eq20}
   \end{align}
    
\end{frame}


\begin{frame}{}
   Finally substituting equations \eqref{eq11} and \eqref{eq20} in equation \eqref{eq9} we get,
   \begin{align}
     E\cbrak{x^2} &= n \, E\cbrak{y^2} \, E\cbrak{\frac{1}{z}} \\
                  &= n \cbrak{1} \cbrak{\frac{1}{n - 2}} \\
                  &= \frac{n}{n - 2}
   \end{align}
   Hence, we have proved the below when the random variable has a student $t$ distribution.
   \begin{align}
      \therefore E\cbrak{x^2} = \frac{n}{n - 2}
   \end{align}
\end{frame}




\end{document}