% Inbuilt themes in beamer
\documentclass{beamer}
% Theme choice:
\usetheme{CambridgeUS}

% Packages 
\usepackage{listings}
\usepackage{blkarray}
\usepackage{listings}
\usepackage{subcaption}
\usepackage{url}
\usepackage{tikz}
\usepackage{tkz-euclide} % loads  TikZ and tkz-base
%\usetkzobj{all}
\usetikzlibrary{calc,math}
\usepackage{float}
\renewcommand{\vec}[1]{\mathbf{#1}}
\usepackage[export]{adjustbox}
\usepackage[utf8]{inputenc}
\usepackage{amsmath}
\usepackage{amsfonts}
\usepackage{tikz}
\usepackage{hyperref}
\usepackage{bm}
\usetikzlibrary{automata, positioning}
\providecommand{\pr}[1]{\ensuremath{\Pr\left(#1\right)}}
\providecommand{\mbf}{\mathbf}
\providecommand{\qfunc}[1]{\ensuremath{Q\left(#1\right)}}
\providecommand{\sbrak}[1]{\ensuremath{{}\left[#1\right]}}
\providecommand{\lsbrak}[1]{\ensuremath{{}\left[#1\right.}}
\providecommand{\rsbrak}[1]{\ensuremath{{}\left.#1\right]}}
\providecommand{\brak}[1]{\ensuremath{\left(#1\right)}}
\providecommand{\lbrak}[1]{\ensuremath{\left(#1\right.}}
\providecommand{\rbrak}[1]{\ensuremath{\left.#1\right)}}
\providecommand{\cbrak}[1]{\ensuremath{\left\{#1\right\}}}
\providecommand{\lcbrak}[1]{\ensuremath{\left\{#1\right.}}
\providecommand{\rcbrak}[1]{\ensuremath{\left.#1\right\}}}
\providecommand{\abs}[1]{\vert#1\vert}

\newcounter{saveenumi}
\newcommand{\seti}{\setcounter{saveenumi}{\value{enumi}}}
\newcommand{\conti}{\setcounter{enumi}{\value{saveenumi}}}
\usepackage{amsmath}
\setbeamertemplate{caption}[numbered]{}


% Title page details: 
\title{AI1110 \\ Assignment-7} 
\author{R Bhargava Ram \\ CS21BTECH11052}
\date{\today}
\logo{\large \LaTeX{}}


\begin{document}

% Title page frame
\begin{frame}
    \titlepage 
\end{frame}

% Remove logo from the next slides
\logo{}


% Outline frame
\begin{frame}{Outline}
    \tableofcontents
\end{frame}

% Lists frame
\section{Abstract}
	\begin{frame}{Abstract}
		\begin{itemize}
			\item This document contains the solution to a Question
			\item In NCERT Class 12 Textbook
			\item In Chapter 13 (Probability)
		\end{itemize}
		
	\end{frame}


\section{Question}
    \begin{frame}{Question}
       \begin{block}{\textbf{Exercise 13.2.7}}
 Given that the events A and B are such that $\pr{A} = \frac{1}{2}$, $\pr{A \cup B} = \frac{3}{5}$ and $\pr{B} = p$. Find $p$ if they are
\begin{enumerate}
    \item  mutually exclusive  
    \item independent. 
\end{enumerate}
       \end{block}
         
    \end{frame}


\section{Theory}
	\begin{frame}{Theory}
			 \begin{block}{Inclusion-Exclusion Principle}
			    If A and B are two events, the individual probabilities and probability of occurrence of both events at same time are known,then probability of occurrence of either event A or B is given as
			    \begin{align}
				    \pr{A+B} = \pr{A} + \pr{B} - \pr{AB}
				\end{align}	
			\end{block}
			
			\begin{block}{Mutually Exclusive Events}
			   Two events A and B are said to be mutually exclusive if they cannot occur at the same time or simultaneously. Mutually exclusive events are also called Disjoint events. So,
			    \begin{align}
				    \pr{AB} = 0
			    \end{align}
		    \end{block}	
		    
	\end{frame}
	
	\begin{frame}	     
		    \begin{block}{Independent Events}
			   Two events A and B are said to be independent events if the probability of occurrence of one of them is not affected by occurrence of the other.
			   \begin{align}
				   \pr{AB}=\pr{A} \times \pr{B}
			   \end{align}
		    \end{block}  
		
	\end{frame}
	
	
\section{Solution}	
	\begin{frame}{Solution}
		We have,
		      \begin{align}
                     \pr{A} &= \frac{1}{2} \\
                 \pr{A + B} &= \frac{3}{5} \\
                     \pr{B} &= p
              \end{align}
		  
		let, $\pr{AB} = x$
		      
	    Using Inclusion-Exclusion Principle and substituting,
	          \begin{align}
                    \pr{A + B} &= \pr{A} + \pr{B} - \pr{AB} \\
                   \frac{3}{5} &= \frac{1}{2} + p - x \\
                \implies p - x &= \frac{3}{5} - \frac{1}{2} \\
                     \label{eq:condition}
                \implies p - x &= \frac{1}{10}
              \end{align}
              
	\end{frame}

    \begin{frame}     
        \begin{enumerate}
           \item When events A and B are mutually exclusive,
           
             From definition,
               \begin{align}
                  \pr{AB} &= 0 \\
                     \label{eq:1}
                        x &= 0
               \end{align}
               
            On substituting \eqref{eq:1} in equation \eqref{eq:condition},
              \begin{align}
                   p - 0 &= \frac{1}{10} \\ 
                       p &= \frac{1}{10}         
              \end{align}
        \seti            
        \end{enumerate}
                              
    \end{frame} 
    
    \begin{frame}
       \begin{enumerate}
          \conti
          \item When events A and B are independent,
          
            From definition,
               \begin{align}
                  \pr{AB} &= \pr{A} \times \pr{B} \\
                        x &= \frac{1}{2} \times p \\
                      \label{eq:2}
                        x &= \frac{p}{2}
               \end{align}
            On substituting \eqref{eq:2} in equation \eqref{eq:condition},
            \begin{align}
                p - \frac{p}{2} &= \frac{1}{10} \\ 
                    \frac{p}{2} &= \frac{1}{10} \\
                              p &= \frac{1}{5}
            \end{align} 
       \end{enumerate}

    \end{frame}


\end{document}