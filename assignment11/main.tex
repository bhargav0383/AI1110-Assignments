%%%%%%%%%%%%%%%%%%%%%%%%%%%%%%%%%%%%%%%%%%%%%%%%%%%%%%%%%%%%%%%
%	
% Welcome to Overleaf --- just edit your LaTeX on the left,
% and we'll compile it for you on the right. If you open the
% 'Share' menu, you can invite other users to edit at the same
% time. See www.overleaf.com/learn for more info. Enjoy!
%
%%%%%%%%%%%%%%%%%%%%%%%%%%%%%%%%%%%%%%%%%%%%%%%%%%%%%%%%%%%%%%%

% Inbuilt themes in beamer
\documentclass{beamer}

%packages:
% \usepackage{tfrupee}
% \usepackage{amsmath}
% \usepackage{amssymb}
% \usepackage{gensymb}
% \usepackage{txfonts}

% \def\inputGnumericTable{}

% \usepackage[latin1]{inputenc}                                 
% \usepackage{color}                                            
% \usepackage{array}                                            
% \usepackage{longtable}                                        
% \usepackage{calc}                                             
% \usepackage{multirow}                                         
% \usepackage{hhline}                                           
% \usepackage{ifthen}
% \usepackage{caption} 
% \captionsetup[table]{skip=3pt}  
% \providecommand{\pr}[1]{\ensuremath{\Pr\left(#1\right)}}
% \providecommand{\cbrak}[1]{\ensuremath{\left\{#1\right\}}}
% %\renewcommand{\thefigure}{\arabic{table}}
% \renewcommand{\thetable}{\arabic{table}}      

\setbeamertemplate{caption}[numbered]{}

\usepackage{enumitem}
\usepackage{tfrupee}
\usepackage{amsmath}
\usepackage{amssymb}
\usepackage{gensymb}
\usepackage{graphicx}
\usepackage{txfonts}

\def\inputGnumericTable{}

\usepackage[latin1]{inputenc}                                 
\usepackage{color}    
\usepackage{textcomp, gensymb}         
\usepackage{array}                                            
\usepackage{longtable}                                        
\usepackage{calc}                                             
\usepackage{multirow}                                         
\usepackage{hhline}                             
\usepackage{mathtools}
\usepackage{ifthen}
\usepackage{caption} 
\providecommand{\pr}[1]{\ensuremath{\Pr\left(#1\right)}}
\providecommand{\cbrak}[1]{\ensuremath{\left\{#1\right\}}}
\renewcommand{\thefigure}{\arabic{table}}
\renewcommand{\thetable}{\arabic{table}}   
\providecommand{\brak}[1]{\ensuremath{\left(#1\right)}}

% Theme choice:
\usetheme{CambridgeUS}

% Title page details: 
\title{AI1110 \\ Assignment-11} 
\author{Rajulapati Bhargava Ram \\ CS21BTECH11052}
\date{\today}
\logo{\large \LaTeX{}}


\begin{document}

% Title page frame
\begin{frame}
    \titlepage 
\end{frame}
\logo{}


% Outline frame
\begin{frame}{Outline}
    \tableofcontents
\end{frame}



\section{Question}
\begin{frame}{Question}
    \begin{block}{\textbf{Papoullis 8-17:} } 
      Suppose that the IQ scores of children in a certain grade are the samples of an $N(\eta, \sigma)$ random variable $x$. We test 10 children and obtain the following averages: $\overline{x}=90$, $s=5$. Find the 0.95 confidence interval of $\eta$ and of $\sigma$.       
     \end{block}
     
\end{frame}


\section{Theory}
\begin{frame}{Theory}
   \begin{block}{UNKNOWN VARIANCE}
      If $\sigma$ is unknown, we form the sample variance to estimate $\eta$
     \begin{align}
       s^2 = \frac{1}{n-1} \sum_{i=1}^{n} (x_i - \overline{x})^2 
     \end{align}
      This is a unbiased estimate of $\sigma^2$ and it tends to $\sigma^2$ as $n \rightarrow \infty$. 
      \end{block}
      
      \begin{block}{}
      If $x$ is normal, the ratio
      \begin{align}
         \frac{\overline{x} - \eta}{s/\sqrt{n}}
      \end{align}
      has a Student $t$ distribution with $n-1$ degrees of freedom.  
  \end{block}   
  
	
\end{frame}

\begin{frame}
   \begin{block}{}
      Denoting by $t_u$ its $u$ percentiles $(u=1-\delta)$. This yields the interval, 
      \begin{align}
        \overline{x} - t_{1- \delta/2} \frac{s}{\sqrt{n}} < \eta <  \overline{x} + t_{1- \delta/2} \frac{s}{\sqrt{n}}   \label{eq3}
	  \end{align} 
	        
   \end{block}
   
   \begin{block}{UNKNOWN MEAN}
      If $\eta$ is unknown, we use as the point estimate of $\sigma^2$ the sample variance $s^2$. The random variable $(n-1) s^2/\sigma^2$ has a $\chi^2 (n-1)$ distribution. This yields the interval,
      \begin{align}
        \frac{(n-1) s^2}{{\chi^2}_{1- \delta/2} (n-1)} < \sigma^2 < \frac{(n-1) s^2}{{\chi^2}_{\delta/2} (n-1)}  \label{eq4}
      \end{align}
 
   \end{block}
   
\end{frame}

\section{Solution}
\begin{frame}{Solution}
	Given,
	
	  Sample size, $n = 10$,
	  
      Sample mean, defined as
      \begin{align}
         \overline{x} = \frac{1}{n} \sum_{i=1}^{n} x_i
      \end{align}
       from question, $\overline{x} = 90$ 
	  
	 Sample standard deviation, $s = 5$.

      
\end{frame}


\begin{frame}{The 0.95 confidence interval of $\eta$}
    Using equation \eqref{eq3}, 
   \begin{align}
      \overline{x} - t_{0.975}(9) \: \frac{s}{\sqrt{n}} < \eta <  \overline{x} + t_{0.975}(9) \: \frac{s}{\sqrt{n}} 
   \end{align}
   From Table 8-2 in papoullis book we get the value of $t_{0.975}(9) = 2.26$,
  \begin{align}
    t_{0.975}(9) \: \frac{s}{\sqrt{n}} &= (2.26)\frac{5}{\sqrt{10}} \\
      								  &= 3.57
  \end{align}
  
   On substituting the above we get,
   \begin{align}
      \overline{x} - 3.57 &< \eta < \overline{x} + 3.57 \\
      90 - 3.57 &< \eta < 90 + 3.57 \\
     \therefore 86.43 &< \eta < 93.57			  
   \end{align}
   
       
\end{frame}
 

\begin{frame}{The 0.95 confidence interval of $\sigma$}
   Using equation \eqref{eq4},
   \begin{align}
      \frac{(n-1) s^2}{{\chi^2}_{1- \delta/2} (n-1)} &< \sigma^2 < \frac{(n-1) s^2}{{\chi^2}_{\delta/2} (n-1)} \\
      \frac{9 \times 5^2}{{\chi^2}_{0.975} (9)} &< \sigma^2 < \frac{9 \times 5^2}{{\chi^2}_{0.025} (9)}
   \end{align}  
     
   From Table 8-3 in papoullis book we get the value of ${\chi^2}_{0.975} (9) = 19.02$ and ${\chi^2}_{0.025} (9) = 2.70$,
   \begin{align}
       \frac{9 \times 5^2}{19.02} &< \sigma^2 < \frac{9 \times 5^2}{2.70} \\
       11.83 &< \sigma^2 < 83.33 \\
     \therefore  3.44 &< \sigma < 9.13
   \end{align}
    
\end{frame}



\end{document}