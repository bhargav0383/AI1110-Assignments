\let\negmedspace\undefined
\let\negthickspace\undefined
%\RequirePackage{amsmath}
\documentclass[journal,12pt,twocolumn]{IEEEtran}
%
% \usepackage{setspace}
 \usepackage{gensymb}
  \usepackage[misc]{ifsym}
%\doublespacing
 \usepackage{polynom}
%\singlespacing
%\usepackage{silence}
%Disable all warnings issued by latex starting with "You have..."
%\usepackage{graphicx}
\usepackage{amssymb}
%\usepackage{relsize}
\usepackage[cmex10]{amsmath}
%\usepackage{amsthm}
%\interdisplaylinepenalty=2500
%\savesymbol{iint}
%\usepackage{txfonts}
%\restoresymbol{TXF}{iint}
%\usepackage{wasysym}
\usepackage{amsthm}
%\usepackage{pifont}
%\usepackage{iithtlc}
% \usepackage{mathrsfs}
% \usepackage{txfonts}
 \usepackage{stfloats}
% \usepackage{steinmetz}
 \usepackage{bm}
% \usepackage{cite}
% \usepackage{cases}
% \usepackage{subfig}
%\usepackage{xtab}
\usepackage{longtable}
%\usepackage{multirow}
%\usepackage{algorithm}
%\usepackage{algpseudocode}
\usepackage{enumitem}
 \usepackage{mathtools}
 \usepackage{tikz}
% \usepackage{circuitikz}
% \usepackage{verbatim}
%\usepackage{tfrupee}
\usepackage[breaklinks=true]{hyperref}
%\usepackage{stmaryrd}
%\usepackage{tkz-euclide} % loads  TikZ and tkz-base
%\usetkzobj{all}
\usepackage{listings}
    \usepackage{color}                                            %%
    \usepackage{array}                                            %%
    \usepackage{longtable}                                        %%
    \usepackage{calc}                                             %%
    \usepackage{multirow}                                         %%
    \usepackage{hhline}                                           %%
    \usepackage{ifthen}                                           %%
  %optionally (for landscape tables embedded in another document): %%
    \usepackage{lscape}     
% \usepackage{multicol}
% \usepackage{chngcntr}
%\usepackage{enumerate}
\usepackage{tfrupee}

%\usepackage{wasysym}
%\newcounter{MYtempeqncnt}
\DeclareMathOperator*{\Res}{Res}
\DeclareMathOperator*{\equals}{=}
%\renewcommand{\baselinestretch}{2}
%\renewcommand\thesection{\arabic{section}}
%\renewcommand\thesubsection{\thesection.\arabic{subsection}}
%\renewcommand\thesubsubsection{\thesubsection.\arabic{subsubsection}}

%\renewcommand\thesectiondis{\arabic{section}}
%\renewcommand\thesubsectiondis{\thesectiondis.\arabic{subsection}}
%\renewcommand\thesubsubsectiondis{\thesubsectiondis.\arabic{subsubsection}}

% correct bad hyphenation here
\hyphenation{op-tical net-works semi-conduc-tor}
\def\inputGnumericTable{}                                 %%

\lstset{
%language=C,
frame=single, 
breaklines=true,
columns=fullflexible
}
%\lstset{
%language=tex,
%frame=single, 
%breaklines=true
%}
\begin{document}

%


\newtheorem{theorem}{Theorem}[section]
\newtheorem{problem}{Problem}
\newtheorem{proposition}{Proposition}[section]
\newtheorem{lemma}{Lemma}[section]
\newtheorem{corollary}[theorem]{Corollary}
\newtheorem{example}{Example}[section]
\newtheorem{definition}[problem]{Definition}
%\newtheorem{thm}{Theorem}[section] 
%\newtheorem{defn}[thm]{Definition}
%\newtheorem{algorithm}{Algorithm}[section]
%\newtheorem{cor}{Corollary}
\newcommand{\BEQA}{\begin{eqnarray}}
\newcommand{\EEQA}{\end{eqnarray}}
\newcommand{\define}{\stackrel{\triangle}{=}}
\newcommand*\circled[1]{\tikz[baseline=(char.base)]{
    \node[shape=circle,draw,inner sep=2pt] (char) {#1};}}
\bibliographystyle{IEEEtran}
%\bibliographystyle{ieeetr}
\providecommand{\mbf}{\mathbf}
\providecommand{\pr}[1]{\ensuremath{\Pr\left(#1\right)}}
\providecommand{\qfunc}[1]{\ensuremath{Q\left(#1\right)}}
\providecommand{\sbrak}[1]{\ensuremath{{}\left[#1\right]}}
\providecommand{\lsbrak}[1]{\ensuremath{{}\left[#1\right.}}
\providecommand{\rsbrak}[1]{\ensuremath{{}\left.#1\right]}}
\providecommand{\brak}[1]{\ensuremath{\left(#1\right)}}
\providecommand{\lbrak}[1]{\ensuremath{\left(#1\right.}}
\providecommand{\rbrak}[1]{\ensuremath{\left.#1\right)}}
\providecommand{\cbrak}[1]{\ensuremath{\left\{#1\right\}}}
\providecommand{\lcbrak}[1]{\ensuremath{\left\{#1\right.}}
\providecommand{\rcbrak}[1]{\ensuremath{\left.#1\right\}}}
\theoremstyle{remark}
\newtheorem{rem}{Remark}
\newcommand{\sgn}{\mathop{\mathrm{sgn}}}
\providecommand{\fourier}{\overset{\mathcal{F}}{ \rightleftharpoons}}
%\providecommand{\hilbert}{\overset{\mathcal{H}}{ \rightleftharpoons}}
\providecommand{\system}{\overset{\mathcal{H}}{ \longleftrightarrow}}
	%\newcommand{\solution}[2]{\textbf{Solution:}{#1}}
\newcommand{\solution}{\noindent \textbf{Solution: }}
\newcommand{\cosec}{\,\text{cosec}\,}
\providecommand{\dec}[2]{\ensuremath{\overset{#1}{\underset{#2}{\gtrless}}}}
\newcommand{\myvec}[1]{\ensuremath{\begin{pmatrix}#1\end{pmatrix}}}
\newcommand{\mydet}[1]{\ensuremath{\begin{vmatrix}#1\end{vmatrix}}}
%\numberwithin{equation}{section}
%\numberwithin{figure}{section}
%\numberwithin{table}{section}
%\numberwithin{equation}{subsection}
%\numberwithin{problem}{section}
%\numberwithin{definition}{section}
\makeatletter
\@addtoreset{figure}{problem}
\makeatother
\let\StandardTheFigure\thefigure
\let\vec\mathbf
%\renewcommand{\thefigure}{\theproblem.\arabic{figure}}
\renewcommand{\thefigure}{\theproblem}
%\setlist[enumerate,1]{before=\renewcommand\theequation{\theenumi.\arabic{equation}}
%\counterwithin{equation}{enumi}
%\renewcommand{\theequation}{\arabic{subsection}.\arabic{equation}}
\def\putbox#1#2#3{\makebox[0in][l]{\makebox[#1][l]{}\raisebox{\baselineskip}[0in][0in]{\raisebox{#2}[0in][0in]{#3}}}}
     \def\rightbox#1{\makebox[0in][r]{#1}}
     \def\centbox#1{\makebox[0in]{#1}}
     \def\topbox#1{\raisebox{-\baselineskip}[0in][0in]{#1}}
     \def\midbox#1{\raisebox{-0.5\baselineskip}[0in][0in]{#1}}
\title{
	%\logo{
%Computational Approach to School Geometry
	Assignment
%	}
}
\author{ R Bhargava Ram\\CS21BTECH11052% <-this % stops a space
}	
%\title{
%	\logo{Matrix Analysis through Octave}{\begin{center}\includegraphics[scale=.24]{tlc}\end{center}}{}{HAMDSP}
%}
% paper title
% can use linebreaks \\ within to get better formatting as desired
%\title{Matrix Analysis through Octave}
%
%
% author names and IEEE memberships
% note positions of commas and nonbreaking spaces ( ~ ) LaTeX will not break
% a structure at a ~ so this keeps an author's name from being broken across
% two lines.
% use \thanks{} to gain access to the first footnote area
% a separate \thanks must be used for each paragraph as LaTeX2e's \thanks
% was not built to handle multiple paragraphs
%
%\author{<-this % stops a space
%\thanks{}}
%}
% note the % following the last \IEEEmembership and also \thanks - 
% these prevent an unwanted space from occurring between the last author name
% and the end of the author line. i.e., if you had this:
% 
% \author{....lastname \thanks{...} \thanks{...} }
%                     ^------------^------------^----Do not want these spaces!
%
% a space would be appended to the last name and could cause every name on that
% line to be shifted left slightly. This is one of those "LaTeX things". For
% instance, "\textbf{A} \textbf{B}" will typeset as "A B" not "AB". To get
% "AB" then you have to do: "\textbf{A}\textbf{B}"
% \thanks is no different in this regard, so shield the last } of each \thanks
% that ends a line with a % and do not let a space in before the next \thanks.
% Spaces after \IEEEmembership other than the last one are OK (and needed) as
% you are supposed to have spaces between the names. For what it is worth,
% this is a minor point as most people would not even notice if the said evil
% space somehow managed to creep in.
%\WarningFilter{latex}{LaTeX Warning: You have requested, on input line 117, version}
% The paper headers
%\markboth{Journal of \LaTeX\ Class Files,~Vol.~6, No.~1, January~2007}%
%{Shell \MakeLowercase{\textit{et al.}}: Bare Demo of IEEEtran.cls for Journals}
% The only time the second header will appear is for the odd numbered pages
% after the title page when using the twoside option.
% 
% *** Note that you probably will NOT want to include the author's ***
% *** name in the headers of peer review papers.                   ***
% You can use \ifCLASSOPTIONpeerreview for conditional compilation here if
% you desire.
% If you want to put a publisher's ID mark on the page you can do it like
% this:
%\IEEEpubid{0000--0000/00\$00.00~\copyright~2007 IEEE}
% Remember, if you use this you must call \IEEEpubidadjcol in the second
% column for its text to clear the IEEEpubid mark.
% make the title area
\maketitle
\tableofcontents
\bigskip
\renewcommand{\thefigure}{\theenumi}
\renewcommand{\thetable}{\theenumi}
%\renewcommand{\theequation}{\theenumi}
%\begin{abstract}
%%\boldmath
%In this letter, an algorithm for evaluating the exact analytical bit error rate  (BER)  for the piecewise linear (PL) combiner for  multiple relays is presented. Previous results were available only for upto three relays. The algorithm is unique in the sense that  the actual mathematical expressions, that are prohibitively large, need not be explicitly obtained. The diversity gain due to multiple relays is shown through plots of the analytical BER, well supported by simulations. 
%
%\end{abstract}
% IEEEtran.cls defaults to using nonbold math in the Abstract.
% This preserves the distinction between vectors and scalars. However,
% if the journal you are submitting to favors bold math in the abstract,
% then you can use LaTeX's standard command \boldmath at the very start
% of the abstract to achieve this. Many IEEE journals frown on math
% in the abstract anyway.
% Note that keywords are not normally used for peerreview papers.
%\begin{IEEEkeywords}
%Cooperative diversity, decode and forward, piecewise linear
%\end{IEEEkeywords}
% For peer review papers, you can put extra information on the cover
% page as needed:
% \ifCLASSOPTIONpeerreview
% \begin{center} \bfseries EDICS Category: 3-BBND \end{center}
% \fi
%
% For peerreview papers, this IEEEtran command inserts a page break and
% creates the second title. It will be ignored for other modes.
%\IEEEpeerreviewmaketitle
\begin{abstract}
This manual provides solutions to the Assignment of Random Numbers
\end{abstract}

%template ends here
\section{Uniform Random Numbers}
Let $U$ be a uniform random variable between 0 and 1.
\begin{enumerate}[label=\thesection.\arabic*
,ref=\thesection.\theenumi]

%%%%%%%%%%%%%%%%%%%%%%%%%%%%%%%%%%%%%%%%
\item Generate $10^6$ samples of $U$ using a C program and save into a file called uni.dat .
\\
\solution Download the following files and execute the  C program.
\begin{lstlisting}
https://github.com/bhargav0383/AI1110-Assignments/blob/main/RandomNumbers/codes/1.1.c
https://github.com/bhargav0383/AI1110-Assignments/blob/main/RandomNumbers/codes/functions.h
\end{lstlisting}
Execute the above C program files using the following commands
\begin{lstlisting}
$ gcc 1.1.c
$ ./a.out
\end{lstlisting}

%%%%%%%%%%%%%%%%%%%%%%%%%%%%%%%%%%%%%%
\item
Load the uni.dat file into python and plot the empirical CDF of $U$ using the samples in uni.dat. The CDF is defined as
\begin{align}
F_{U}(x) = \pr{U \le x}
\end{align}
\\
\solution  The python code for the plot in Fig. \ref{fig:1.2} is given below,
\begin{lstlisting}
https://github.com/bhargav0383/AI1110-Assignments/blob/main/RandomNumbers/codes/1.2CDF.py
\end{lstlisting}
Download the above file and execute the command below to produce Fig.\ref{fig:1.2}
\begin{lstlisting}
$ python3 1.2CDF.py
\end{lstlisting}
\begin{figure}[!h]
\centering
\includegraphics[width=\columnwidth]{./figs/1.2CDF.png}
\caption{The CDF of $U$}
\label{fig:1.2}
\end{figure}
%

%%%%%%%%%%%%%%%%%%%%%%%%%%%%%%%%%%%%%%%%
\item
Find a  theoretical expression for $F_{U}(x)$.\\
\solution Given $U$ is a uniform Random Variable between 0 and 1,
\begin{align}
F_U(x) = \pr{U \leq x} = \int_{-\infty}^{x}p_U(u)du
\end{align}
We have three parts:
		\begin{enumerate}[label=\roman*)]
			\item For $x < 0$; $p_X(x) = 0$, So $F_U(x) = 0$.
			\item For $0 \leq x < 1$;
				\begin{align}
					F_U(x) = \int_{0}^{x}(1)du = x
				\end{align}
			\item For $x \geq 1$; CDF is 1 as all the random numbers are between 0 and 1.
		\end{enumerate}
Therefore,
		\begin{align}
			F_U(x) = 
			\begin{cases}
				0 & x < 0 \\
				x & 0 \leq x < 1 \\
				1 & x \geq 1
			\end{cases}
		\end{align}
		
%%%%%%%%%%%%%%%%%%%%%%%%%%%%%%%%%%%%%%%%%%%%%%%%%
\item
The mean of $U$ is defined as
%
\begin{equation}
E\sbrak{U} = \frac{1}{N}\sum_{i=1}^{N}U_i
\end{equation}
%
and its variance as
%
\begin{equation}
\text{var}\sbrak{U} = E\sbrak{U- E\sbrak{U}}^2 
\end{equation}
Write a C program to  find the mean and variance of $U$. \\
\solution Download the following files and execute the  C program.
\begin{lstlisting}
https://github.com/bhargav0383/AI1110-Assignments/blob/main/RandomNumbers/codes/1.4.c
https://github.com/bhargav0383/AI1110-Assignments/blob/main/RandomNumbers/codes/functions.h
\end{lstlisting}
Execute the above C program files using the following commands
\begin{lstlisting}
$ gcc 1.4.c
$ ./a.out
\end{lstlisting}

%%%%%%%%%%%%%%%%%%%%%%%%%%%%%%%%%%%%%%%%%%%%%%%%%%%%%%%%%%%%%
\item Verify your result theoretically given that
\end{enumerate}
%
\begin{equation}
E\sbrak{U^k} = \int_{-\infty}^{\infty}x^kdF_{U}(x)
\end{equation}
\solution 
W.K.T,
\begin{align}
    \text{var}\sbrak{U} &= E\sbrak{U^2}- E\sbrak{U}^2
\end{align}
Where $E\sbrak{U}$ is,
\begin{align}
   E\sbrak{U}&=\int_{-\infty}^{\infty}xdF_U(x)\\
             &=\int_{0}^{1}x\\
             &=\frac{1}{2} = 0.5
\end{align}
And $E\sbrak{U^2}$ is,
\begin{align}
    E\sbrak{U^2}&=\int_{-\infty}^{\infty}x^{2}dF_U(x)\\
                &=\int_{0}^{1}x^{2}dF_U(x)\\
                &=\frac{1}{3}\\
\end{align}
Hence finally,
\begin{align}
    \implies \boxed{\text{var}\sbrak{U}=\frac{1}{12}=0.0833}
\end{align}

%%%%%%%%%%%%%%%%%%%%%%%%%%%%%%%%%%%%%%%%%%%%%%%%%%%%%%%%%%%%%%%%%
\section{Central Limit Theorem}
%
\begin{enumerate}[label=\thesection.\arabic*
,ref=\thesection.\theenumi]
%
\item
Generate $10^6$ samples of the random variable
%
\begin{equation}
X = \sum_{i=1}^{12}U_i -6
\end{equation}
%
using a C program, where $U_i, i = 1,2,\dots, 12$ are  a set of independent uniform random variables between 0 and 1 and save in a file called gau.dat\\
\solution Download the following files and execute the  C program.
\begin{lstlisting}
https://github.com/bhargav0383/AI1110-Assignments/blob/main/RandomNumbers/codes/2.1.c
https://github.com/bhargav0383/AI1110-Assignments/blob/main/RandomNumbers/codes/functions.h
\end{lstlisting}
Execute the above C program files using the following commands
\begin{lstlisting}
$ gcc 2.1.c
$ ./a.out
\end{lstlisting}

%%%%%%%%%%%%%%%%%%%%%%%%%%%%%%%%%%%%%%%%%%%%%%%%%
\item
Load gau.dat in python and plot the empirical CDF of $X$ using the samples in gau.dat. What properties does a CDF have?\\
\solution The CDF of $X$ is plotted in Fig. \ref{fig:2.2}\\
using the code below
\begin{lstlisting}
https://github.com/bhargav0383/AI1110-Assignments/blob/main/RandomNumbers/codes/2.2CDF.py
\end{lstlisting}
\begin{figure}[!h]
\centering
\includegraphics[width=\columnwidth]{./figs/2.2CDF.png}
\caption{The CDF of $X$}
\label{fig:2.2}
\end{figure}
Download the above file and execute the following command to produce Fig.\ref{fig:2.2}
\begin{lstlisting}
$ python3 2.2CDF.py
\end{lstlisting}
Some of the properties of CDF 
\begin{enumerate}
    \item $F_X(x)$ is non decreasing function.
    \item As $x \to -\infty$, $F_X(x) \to 0$ and when $x \to \infty$, $F_X(x) \to 1$
    \item Graph is linear upto some region around $x=0$.
\end{enumerate}

%%%%%%%%%%%%%%%%%%%%%%%%%%%%%%%%%%%%%%%%%%%%%
\item
Load gau.dat in python and plot the empirical PDF of $X$ using the samples in gau.dat. The PDF of $X$ is defined as
\begin{align}
p_{X}(x) = \frac{d}{dx}F_{X}(x)
\end{align}
What properties does the PDF have?
\\
\solution The PDF of $X$ is plotted in Fig. \ref{fig:2.3} using the code below
\begin{lstlisting}
https://github.com/bhargav0383/AI1110-Assignments/blob/main/RandomNumbers/codes/2.3PDF.py
\end{lstlisting}
Download the above files and execute the following commands to produce Fig.\ref{fig:2.3}
\begin{lstlisting}
$ python3 2.3PDF.py
\end{lstlisting}
\begin{figure}[!h]
\centering
\includegraphics[width=\columnwidth]{./figs/2.3PDF.png}
\caption{The PDF of $X$}
\label{fig:2.3}
\end{figure}
Some of the properties of the PDF:
\begin{enumerate}
    \item Symmetric about $x=\mu$
    \item Area under the PDF graph is unity.
    \item Increasing function for $x<\mu$ and decreasing for $x>\mu$ and attains maximum at $x=\mu$.
\end{enumerate}

%%%%%%%%%%%%%%%%%%%%%%%%%%%%%%%%%%%%%%%%%%%%%%%%%%%%%%%%%%%%%%%%%%%%%%%%%%%
\item Find the mean and variance of $X$ by writing a C program.\\
\solution Download the following files and execute the  C program.
\begin{lstlisting}
https://github.com/bhargav0383/AI1110-Assignments/blob/main/RandomNumbers/codes/2.4.c
https://github.com/bhargav0383/AI1110-Assignments/blob/main/RandomNumbers/codes/functions.h
\end{lstlisting}
Execute the above C program files using the following commands
\begin{lstlisting}
$ gcc 2.4.c
$ ./a.out
\end{lstlisting}

%%%%%%%%%%%%%%%%%%%%%%%%%%%%%%%%%%%%%%%%%%%%%%%%%%%%%%%%%%%%
\item Given that 
\begin{align}
p_{X}(x) = \frac{1}{\sqrt{2\pi}}\exp\brak{-\frac{x^2}{2}}, -\infty < x < \infty,
\end{align}
repeat the above exercise theoretically.
\end{enumerate}
\solution 
CDF is defined as
    \begin{align}
        F_X(x)&=\int_{-\infty}^{\infty}p_X(x)dx\\
        \text{W.K.T,} \quad \boxed{F_X(x)=1}
    \end{align}
Mean is given by
    \begin{align}
        E(x)&=\int_{-\infty}^{\infty}xp_X(x)dx\\\
            &=\frac{1}{\sqrt{2\pi}}\int_{-\infty}^{\infty}x\exp\brak{-\frac{x^2}{2}}dx
    \end{align}
    It is a odd function function, So its value is 0.
    \begin{align}
        \boxed{E(x)=0}
    \end{align}
Variance is given by
    \begin{align}
        \text{var}\sbrak{U}=E(U^2)-(E(U))^2\\
        \therefore \boxed{\text{var}\sbrak{U}=\sqrt{2}}
    \end{align}


%%%%%%%%%%%%%%%%%%%%%%%%%%%%%%%%%%%%%%%%%%%%%%%%%%%%%%%%%
\section{From Uniform to Other}
\begin{enumerate}[label=\thesection.\arabic*
,ref=\thesection.\theenumi]
%
\item
Generate samples of 
%
\begin{equation}
V = -2\ln\brak{1-U}
\end{equation}
%
and plot its CDF.  \\
\solution Download the following files and execute the  C program.
\begin{lstlisting}
https://github.com/bhargav0383/AI1110-Assignments/blob/main/RandomNumbers/codes/3.1.c
https://github.com/bhargav0383/AI1110-Assignments/blob/main/RandomNumbers/codes/functions.h
\end{lstlisting}
Execute the above C program files using the following commands
\begin{lstlisting}
$ gcc 3.1.c -lm
$ ./a.out
\end{lstlisting}
The CDF of $V$ is plotted in Fig. \ref{fig:3.1} using the code below
\begin{lstlisting}
https://github.com/bhargav0383/AI1110-Assignments/blob/main/RandomNumbers/codes/3.1CDF.py
\end{lstlisting}
Download the above files and execute the following commands to produce plot Fig.\ref{fig:3.1}
\begin{lstlisting}
$ python3 3.1CDF.py
\end{lstlisting}
\begin{figure}[!h]
\centering
\includegraphics[width=\columnwidth]{./figs/3.1CDF.png}
\caption{The PDF of $X$}
\label{fig:3.1}
\end{figure}

%%%%%%%%%%%%%%%%%%%%%%%%%%%%%%%%%%%%%%%%%%%%%%%%%%%%%%%%%%%%%5
\item Find a theoretical expression for $F_V(x)$.\\
\solution
If Y = g(X), 

W.K.T,
\begin{align}
  F_Y(y) = F_X(g^{-1}(y))  
\end{align}

\begin{align}
V &= -2\ln{(1-U)} \\
1-U &= e^{\frac{-V}{2}}\\
U &= 1 - e^{\frac{-V}{2}} \\ 
F_V(x) &= F_U(1 - e^{\frac{-x}{2}}) 
\end{align}
 \begin{align}
\implies
  F_V(x)=
  \begin{cases}
   0                         & x < 0 \\
	1 - e^{\frac{-x}{2}} & x \geq 0
	\end{cases}
 \end{align}
%\item
%Generate the Rayleigh distribution from Uniform. Verify your result through graphical plots.
\end{enumerate}
\end{document}
