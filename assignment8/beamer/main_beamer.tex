% Inbuilt themes in beamer
\documentclass{beamer}
% Theme choice:
\usetheme{CambridgeUS}

% Packages 
\usepackage{listings}
\usepackage{blkarray}
\usepackage{listings}
\usepackage{subcaption}
\usepackage{url}
\usepackage{tikz}
\usepackage{tkz-euclide} % loads  TikZ and tkz-base
%\usetkzobj{all}
\usetikzlibrary{calc,math}
\usepackage{float}
\renewcommand{\vec}[1]{\mathbf{#1}}
\usepackage[export]{adjustbox}
\usepackage[utf8]{inputenc}
\usepackage{amsmath}
\usepackage{amsfonts}
\usepackage{tikz}
\usepackage{hyperref}
\usepackage{bm}
\usetikzlibrary{automata, positioning}
\providecommand{\pr}[1]{\ensuremath{\Pr\left(#1\right)}}
\providecommand{\mbf}{\mathbf}
\providecommand{\qfunc}[1]{\ensuremath{Q\left(#1\right)}}
\providecommand{\sbrak}[1]{\ensuremath{{}\left[#1\right]}}
\providecommand{\lsbrak}[1]{\ensuremath{{}\left[#1\right.}}
\providecommand{\rsbrak}[1]{\ensuremath{{}\left.#1\right]}}
\providecommand{\brak}[1]{\ensuremath{\left(#1\right)}}
\providecommand{\lbrak}[1]{\ensuremath{\left(#1\right.}}
\providecommand{\rbrak}[1]{\ensuremath{\left.#1\right)}}
\providecommand{\cbrak}[1]{\ensuremath{\left\{#1\right\}}}
\providecommand{\lcbrak}[1]{\ensuremath{\left\{#1\right.}}
\providecommand{\rcbrak}[1]{\ensuremath{\left.#1\right\}}}
\providecommand{\abs}[1]{\vert#1\vert}

\newcounter{saveenumi}
\newcommand{\seti}{\setcounter{saveenumi}{\value{enumi}}}
\newcommand{\conti}{\setcounter{enumi}{\value{saveenumi}}}
\usepackage{amsmath}
\setbeamertemplate{caption}[numbered]{}


% Title page details: 
\title{AI1110 \\ Assignment-8} 
\author{R Bhargava Ram \\ CS21BTECH11052}
\date{\today}
\logo{\large \LaTeX{}}


\begin{document}

% Title page frame
\begin{frame}
    \titlepage 
\end{frame}

% Remove logo from the next slides
\logo{}


% Outline frame
\begin{frame}{Outline}
    \tableofcontents
\end{frame}

% Lists frame
\section{Abstract}
	\begin{frame}{Abstract}
		\begin{itemize}
			\item This document contains the solution to a Question
			\item In Papoulis Probability Textbook
			\item In Chapter 2 Problems
		\end{itemize}
		
	\end{frame}


\section{Question}
    \begin{frame}{Question}
       \begin{block}{\textbf{ Problem 2-14:}}
 			The events A and B are mutually exclusive. Can they be independent?
       \end{block}
         
    \end{frame}


\section{Theory}
	\begin{frame}{Theory}
			\begin{block}{Mutually Exclusive Events}
			   Two events A and B are said to be mutually exclusive if they cannot occur at the same time or simultaneously. Mutually exclusive events are also called Disjoint events. So,
			    \begin{align}
				    \pr{AB} = 0
			    \end{align}
		    \end{block}	
		    
	     
		    \begin{block}{Independent Events}
			   Two events A and B are said to be independent events if the probability of occurrence of one of them is not affected by occurrence of the other.
			   \begin{align}
				   \pr{AB}=\pr{A} \times \pr{B}
			   \end{align}
		    \end{block}  
		
	\end{frame}
	
	
\section{Solution}	
	\begin{frame}{Solution}
		Given,
		
		Events A and B are mutually exclusive.
		
		According to question,
		
		If events A and B are also independent then from definitions we 		get
		\begin{align}
   			 \therefore \pr{A} \times \pr{B} = 0
		\end{align}

So, $\pr{A} = 0$ or $\pr{B} = 0$ 

Hence, Events A and B are mutually exclusive and independent if and only if the probability of atleast one of them is 0.
	\end{frame}

   
\end{document}