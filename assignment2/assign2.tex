\documentclass[journal,12pt,twocolumn]{IEEEtran}
\usepackage{amsthm}
\allowbreak
\usepackage{setspace}
\usepackage{gensymb}
\singlespacing
\usepackage[cmex10]{amsmath}
\usepackage{caption}
\usepackage{amsthm}
\usepackage{float}

\DeclareUnicodeCharacter{2212}{-}
\usepackage{tikz}
\usepackage{pgfplots}

\usepackage{mathrsfs}
\usepackage{txfonts}
\usepackage{stfloats}
\usepackage{bm}
\usepackage{cite}
\usepackage{cases}
\usepackage{subfig}

\usepackage{longtable}
\usepackage{multirow}

\usepackage{enumitem}
\usepackage{mathtools}
\usepackage{steinmetz}
\usepackage{tikz}
\usepackage{circuitikz}
\usepackage{verbatim}
\usepackage{tfrupee}
\usepackage[breaklinks=true]{hyperref}
\usepackage{graphicx}
\usepackage{tkz-euclide}
\graphicspath{ {./images/} }
\usetikzlibrary{calc,math}
\usepackage{listings}
\usepackage{color}                                            %%
\usepackage{array}                                            %%
\usepackage{longtable}                                        %%
\usepackage{calc}                                             %%
\usepackage{multirow}                                         %%
\usepackage{hhline}                                           %%
\usepackage{ifthen}                                           %%
\usepackage{lscape}     
\usepackage{multicol}
\usepackage{chngcntr}

\DeclareMathOperator*{\Res}{Res}



\hyphenation{op-tical net-works semi-conduc-tor}
\def\inputGnumericTable{}                                 %%

\lstset{
	%language=C,
	frame=single, 
	breaklines=true,
	columns=fullflexible
}

\begin{document}
	
	\newcommand{\BEQA}{\begin{eqnarray}}
		\newcommand{\EEQA}{\end{eqnarray}}
	\newcommand{\define}{\stackrel{\triangle}{=}}
	\bibliographystyle{IEEEtran}
	\raggedbottom
	\setlength{\parindent}{0pt}
	\providecommand{\mbf}{\mathbf}
	\providecommand{\pr}[1]{\ensuremath{\Pr\left(#1\right)}}
	\providecommand{\qfunc}[1]{\ensuremath{Q\left(#1\right)}}
	\providecommand{\sbrak}[1]{\ensuremath{{}\left[#1\right]}}
	\providecommand{\lsbrak}[1]{\ensuremath{{}\left[#1\right.}}
	\providecommand{\rsbrak}[1]{\ensuremath{{}\left.#1\right]}}
	\providecommand{\brak}[1]{\ensuremath{\left(#1\right)}}
	\providecommand{\lbrak}[1]{\ensuremath{\left(#1\right.}}
	\providecommand{\rbrak}[1]{\ensuremath{\left.#1\right)}}
	\providecommand{\cbrak}[1]{\ensuremath{\left\{#1\right\}}}
	\providecommand{\lcbrak}[1]{\ensuremath{\left\{#1\right.}}
	\providecommand{\rcbrak}[1]{\ensuremath{\left.#1\right\}}}
	\theoremstyle{remark}
	\newtheorem{rem}{Remark}
	\newcommand{\sgn}{\mathop{\mathrm{sgn}}}
	\providecommand{\abs}[1]{$\left\vert#1\right\vert$}
	\providecommand{\res}[1]{\Res\displaylimits_{#1}} 
	\providecommand{\norm}[1]{$\left\lVert#1\right\rVert$}
	%\providecommand{\norm}[1]{\lVert#1\rVert}
	\providecommand{\mtx}[1]{\mathbf{#1}}
	\providecommand{\mean}[1]{E$\left[ #1 \right]$}
	\providecommand{\fourier}{\overset{\mathcal{F}}{ \rightleftharpoons}}
	%\providecommand{\hilbert}{\overset{\mathcal{H}}{ \rightleftharpoons}}
	\providecommand{\system}{\overset{\mathcal{H}}{ \longleftrightarrow}}
	%\newcommand{\solution}[2]{\textbf{Solution:}{#1}}
	\newcommand{\solution}{\noindent \textbf{Solution: }}
	\newcommand{\cosec}{\,\text{cosec}\,}
	\providecommand{\dec}[2]{\ensuremath{\overset{#1}{\underset{#2}{\gtrless}}}}
	\newcommand{\myvec}[1]{\ensuremath{\begin{pmatrix}#1\end{pmatrix}}}
	\newcommand{\mydet}[1]{\ensuremath{\begin{vmatrix}#1\end{vmatrix}}}
	\makeatletter
	\makeatother
	\let\StandardTheFigure\thefigure
	\let\vec\mathbf
	
	\vspace{3cm}
    \title{AI1110 Assignment2}
    \author{Bhargava Ram Rajulapati - CS21BTECH11052}
    \maketitle
	\newpage
	\bigskip
	\renewcommand{\thefigure}{\theenumi}
	\renewcommand{\thetable}{\theenumi}
	
 \textbf{ICSE class 12 paper 2018:}
 \section*{question 19(a)} 
   Given the total cost function for $x$ units of a commodity as:
   \begin{align*}
      C\brak{x} = \frac{1}{3}x^3 + 3x^2 - 16x +2
   \end{align*}
   Find:
   \begin{enumerate}[ label=(\roman*)]
    \item Marginal cost function
    \item Average cost function
   \end{enumerate}
   
  \section*{solution}
     \begin{enumerate}[ label=(\roman*)]
    \item Marginal cost is the change in total cost that arises when the quantity produced is changed by one unit. \\
          Marginal cost function is the derivative of total cost function $C\brak{x}$.\\
          To find the marginal cost(M.C),derive the total cost function to find $C^{\prime}\brak{x}$ or $\frac{dC}{dx}$. \\\\
          
      Given,
          \begin{align}
            C\brak{x} = \frac{1}{3}x^3 + 3x^2 - 16x +2
          \end{align}
          
      On differentiating,
          \begin{align}
            M.C\brak{x} &= \frac{d}{dx} [C\brak{x}]
            \\
            &= \frac{d}{dx} [\frac{1}{3}x^3 + 3x^2 - 16x +2] 
            \\
            &= \frac{1}{3}\brak{3x^2} + 3\brak{2x}-16\brak{1}+0 
            \\
            &= x^2+6x-16
          \end{align}  
          
      Therefore the Marginal cost function is,
          \begin{align*}
            M.C\brak{x}=x^2+6x-16
          \end{align*}        
      
    \item Average cost function is ratio of total cost of items to the number of items or Quantity,
         \begin{align}
           A.C\brak{x}=\frac{C\brak{x}}{Q}
         \end{align}
         
    Here Q is the quantity produced which is $x$ units.
        \begin{align}
          A.C\brak{x} &=\frac{C\brak{x}}{x} 
          \\
          &=\frac{\frac{1}{3}x^3 + 3x^2 - 16x +2}{x} 
          \\
          &=\frac{1}{3}x^2 + 3x - 16 +\frac{2}{x}
        \end{align}    
                  
    Therefore the Average cost function is,
         \begin{align*}
           A.C\brak{x}=\frac{1}{3}x^2 + 3x - 16 +\frac{2}{x}
         \end{align*}
         
   \end{enumerate}
   
\end{document}